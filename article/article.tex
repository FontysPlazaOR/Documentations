\documentclass[11pt]{scrartcl}

\usepackage[english,english]{babel}
\usepackage{cite}
\usepackage{cmbright}
\usepackage[iso,english]{isodate}
\usepackage[T1]{fontenc}
\usepackage[a4paper,portrait,includeall,top=2cm,right=2cm,bottom=2cm,left=3.5cm,marginparsep=0.5cm,marginparwidth=2cm,headheight=1.5cm,headsep=0.5cm,footskip=1.5cm,heightrounded,pdftex,verbose]{geometry}
\usepackage[utf8]{inputenc}
\usepackage{multicol}
\usepackage[singlespacing]{setspace}
\usepackage{wasysym}

%opening
\title{Room Scanner}
\subtitle{An application to show if a room is available.}
\author{Ron Gebauer \and Maximilian Walter}
\date{\today}


\begin{document}

\maketitle
\nocite{*}
\begin{abstract}
	Scanning a QR-Code with room information with a device is possible using the Vuzix M100. Even though the developer community of Vuzix is small and the access is limited, the development for Android with OpenGL ES resulted in a suitable proof of concept for doing so. 
\end{abstract}

\begin{multicols}{2}
	\section{Introduction}
	This article contains information about a small part project of \glqq Fontys Plaza\grqq. \glqq Fontys Plaza\grqq\ is a project of the module GRAP at the Fontys University of Applied Science in Venlo, whose content is the creation of virtual or augmented realities.
	
	This small part deals with the extension of reality by information on various classrooms. It happens all too often that a teacher searches for a free classroom to do some work but this room is often occupied 10 minutes later by the next class. Consequently, the teacher then has to move to the next room, where he might find the same problem. Would not it be a lot easier if the lecturer already sees when and how long a classroom is occupied or free by watching the door.
	This project tried by using the Vuzix M100, a SmartGlass, to solve this problem. For this will a QR-Code, which contains the name or number of the room, be scanned and analyzed to then display a symbol with accompanying information.
	
	\section{Technologies Used}
	Several technologies for virtual and augmented reality are available. All these technologies have different positive aspects and drawbacks. Some of them mentioned in the following subsections.
		\subsection{Google CardBoard}
			The initial idea was to use Googles CardBoard for displaying the Element that indicates whether a room is available or not. Before the start of the development, it occurred that Google CardBoard is good for virtual reality where there is no connection to the outside world. Augmented reality, where still a part of the \emph{real world} is shown is hardly possible with Google CardBoards since the camera is one sided and no overlaying is possible.
	
			Another drawback of the CardBoard is the fact that both eyes are needed to see the 3D Elements.   
		
		\subsection{Vuzix M100}
			The Vuzix M100 Smart Glasses can be used for augmented reality. Using one eye for the \emph{real World} and one eye for the application, it is possible to display additional information to the users enviroment.
		
		\subsection{Android and OpenGL ES}
			The operating system on the smart glasses is Android. Apps are developed in Java. OpenGL Elements are displayed using the OpenGL ES technique.
	
			The Android SDK and the IDE Eclipse are needed to develop for Android.
			
			The Vuzix OS 2, which is based on Android is limited due to the 
	\section{Architecture}
		The architecture relies on mainly three classes. Despite the \emph{MainActivity} that starts the internal \emph{Scanner}-App that reads the QR-Code, The \emph{TimeTableReader} class is needed to give information about the occupation of the rooms. Finaly the \emph{RoomScannerRenderActivity} with a \emph{RoomScannerRenderer} that display the information in form of OpenGL elements on the screen.
		
		The \emph{TimeTableReader} class that reads information from the TimeTable API has been mocked to provide a realistic datas set. 
	
	\section{Discussion}
		The current project is a prototype that reads a QR-Code and based on that displays two different forms. Getting access to the real timetable is needed to put this application to use. Furthermore it is adviseable to extend the Renderer to display information not only about the current state of the room but also to give information about the next hours, days and weeks.
		
		The lack of suitable, up-to-date libraries for augmented reality resulted in the fact that there is no augmented reality but only a Android Device that shows an \linebreak OpenGL Element.  

	\section{Individual Work}
		In this section is explained who edited what.
		\subsection{Max}
			\begin{itemize}
				\item researched different technologies
				\item worked on the inital project structure
				\item created \emph{ElementFree}, the element that is displayed when the room is free.
			\end{itemize}
			 Problems occured in setting up the Vuzix for development. After the reorganisation of the Vuzix-Website not all information have been accessicable. \\
			 Since the OpenGL ES technqiue is very simular to the JOGL and OpenGL exercises in the module \emph{OpenGL}, the drawing itself caused no problem,depite the fact that rendering an image as texture is not possible. Since this was no requirement, no image is used in the current state of the application.  One negative aspect was the fact, that the emulator does not work, therefore the complete developer team had to rely on one Vuzix. This slowed down the development process.\\
			 A big problem was the limited time that was given for the project.  With more time, a nicer in more detailed solution would have been produced.
		\subsection{Ron}
			TODO
			
	\section{Conclusion}
		The use case of the project does not offer much possibilites to extense the skills learned in the OpenGL Course. Even \linebreak though OpenGL elements are used within the application, the hardest part was to the get the Vuzix and its emulator uptodate. Information is rare and a big developer community does not exist for this kind of hardware. \\
		Luckily Vuzix runs Android applications, which can be emulated with every other device. That is why the first steps took place using the smartphone phone emulator provided by Eclipse. \\
\end{multicols}

\bibliographystyle{fontysIEEtranN}
\bibliography{literature}

\section*{Appendix}
needed? QR-Codes ? or class diagram?

\end{document}